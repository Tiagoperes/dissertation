\chapter[Introdução]{Introdução}

A otimização é um campo de suma importância da ciência da computação que procura encontrar a melhor solução para um problema matemático. Em tais problemas, deve-se maximizar ou minimizar alguma característica, por exemplo, ao decidir um caminho entre duas cidades, é desejável minimizar a distância a se percorrer. Infelizmente, a maior parte dos problemas interessantes são também considerados impossíveis de se resolver em tempo razoável, por essa razão, ao invés de procurar pela melhor solução possível, é usualmente preferível aproximá-la através de algoritmos mais rápidos. A grande importância da pesquisa em otimização está na frequência com a qual esses problemas surgem no dia-a-dia, perguntas como ``qual o menor caminho para atingir um destino'', ``qual o melhor carro para se comprar dado um orçamento'' ou ``qual a forma mais rápida de se executar uma tarefa'' fazem parte da vida da maioria das pessoas e resolvê-las de forma eficiente ainda representa um desafio para a computação.

Uma forma de garantir a obtenção da melhor solução (solução ótima) para um dado problema é por meio da avaliação de todas as soluções possíveis (busca exaustiva). Para a maioria dos problemas interessantes, essa é uma tarefa difícil e inviável. Por exemplo, combinar e testar todas as possibilidades de caminho em um grafo com muitas arestas é um problema NP-difícil, ou seja, não pode ser resolvido em tempo hábil considerando o atual estado da arte da computação. Sendo assim, a pesquisa em otimização também busca desenvolver meios para se encontrar soluções suficientemente próximas da ótima, ou seja, trabalha-se com estratégias de aproximação \cite{GreedyAlgorithms}. Dois dos métodos mais comuns de otimização são os algoritmos gulosos e os algoritmos de busca bio-inspirados. Dentre os bio-inspirados, destacamos os algoritmos genéticos, a otimização por colônia de formigas e a inteligência de enxames (PSO). Os algoritmos gulosos são interessantes para se resolver problemas de otimização mono-objetivo, mas existem otimizações mais complexas, onde se deve considerar mais de uma característica (problemas multiobjetivos). Por exemplo, ao escolher um carro, uma pessoa não se preocupa apenas com o preço, mas também com o desempenho, a durabilidade, o consumo, entre outros fatores. Nesse caso, não basta otimizar um único objetivo, mas encontrar uma opção que apresente uma boa relação custo-benefício considerando todos os fatores avaliados. A maneira mais simples de se contornar esse problema é transformando as diversas métricas em uma única função através de uma média ponderada dos objetivos. Por outro lado, essa estratégia pode não ser a ideal, pois é necessário ter um conhecimento prévio do problema para se decidir a relevância (peso) de cada objetivo. Por essa razão, normalmente deseja-se encontrar todas as soluções que são superiores às demais em pelo menos um dos objetivos. Dessa forma, existirão várias soluções que podem ser classificadas como
melhores, sendo que algumas terão melhor desempenho em uma métrica, enquanto outras se darão melhor em outras.  Esse conjunto de soluções é denominado conjunto Pareto ótimo. Assim, na solução de \acp{PMO}, se faz natural a escolha dos algoritmos de busca que apresentem múltiplas soluções e aproximem, da melhor maneira possível, o conjunto Pareto ótimo \cite{Srinivas1994}.

Muitos problemas da vida real podem tirar proveito da otimização multiobjetivo, fazendo necessária a elaboração de estratégias eficientes para resolvê-los. Vários algoritmos bio-inspirados foram propostos nos últimos anos \cite{Deb2002,Zitzler2002,Deb2014}, tornando a otimização multiobjetivo um campo com alta atividade no ramo da Computação Bioinspirada. O primeiro algoritmo evolutivo multiobjetivo (AEMO) proposto foi o \ac{VEGA} \cite{Schaffer1985}, mas os métodos mais bem consolidados na literatura são o NSGA-II \cite{Deb2002} e o SPEA2 \cite{Zitzler2002}, propostos no início dos anos 2000 e considerados os métodos clássicos da otimização multiobjetivo. Entretanto, mais recentemente, foi levantado que, devido a uma pressão seletiva fraca em direção ao conjunto de soluções ótimas em espaços de alta dimensionalidade, tais \textit{frameworks} não funcionam bem quando aplicados a problemas com muitas funções objetivo (4 ou mais) \cite{Deb2014}. Os problemas com 4 ou mais objetivos são geralmente chamados de \textit{many-objective} e para resolvê-los foram propostas novas abordagens como decomposição, e.g. MOEA/D \cite{Zhang2007}, relações de dominância diferenciadas, e.g. $\varepsilon$-MOEA \cite{Aguirre2009} e evolução baseada em indicadores, e.g. HypE \cite{Bader2011}. Recentemente, novos algoritmos genéticos foram propostos para resolver a problemática \textit{many-objective}, dentre eles destacamos aqui: NSGA-III \cite{Deb2014}, MEAMT \cite{Brasil2013} e MEANDS \cite{Lafeta2017}.

A maior parte dos trabalhos em otimização multiobjetivo utiliza \acp{AG}. Entretanto, outras técnicas bio-inspiradas também podem ser empregadas, tais como: os métodos baseados em colônias de formigas, em inglês \ac{ACO}, e os algoritmos inspirados em inteligência de enxame, em inglês \ac{PSO}. Devido a seus cálculos vetoriais difíceis de se traduzir para um ambiente discreto, os \acp{PSO} são indicados especialmente para problemas contínuos e, portanto, foram deixados de fora deste estudo. Por outro lado, os \acp{ACO} lidam especialmente com problemas discretos e representações em grafos, tornando-os interessantes para os problemas estudados. Dentre os ACOs investigados na literatura para problemas com muitos objetivos, destacam-se o MOACS \cite{Baran2003} e o MOEA/D-ACO \cite{Ke2013}.

Neste trabalho, investigaram-se diversos algoritmos bio-inspirados presentes na literatura de otimização multiobjetivo e propôs-se um novo método de otimização baseado em colônias de formigas (ACO), o \ac{MACO/NDS} \cite{Franca2018}, em português, otimização para muitos objetivos com colônia de formigas baseada em conjuntos de soluções não-dominadas. O algoritmo proposto adota algumas ideias do MEANDS \cite{Lafeta2017}, transferindo-as para uma aplicação em ACO, que tem uma abordagem diferente dos algoritmos genéticos tradicionais, usados na maioria dos métodos de otimização multiobjetivo. A fim de comparar o desempenho dos algoritmos investigados em diferentes \acp{PMO}, utilizaram-se dois problemas discretos bem conhecidos na literatura de otimização multiobjetivo: o \ac{PMM} e o \ac{PRM}. O primeiro é uma versão multiobjetivo do clássico problema da mochila 0/1. O problema da mochila original consiste de uma mochila e um conjunto de itens, no qual deve-se encontrar a melhor combinação de objetos para se colocar na mochila de forma que não se ultrapasse a capacidade da mesma e que se maximize o valor (lucro) dos itens carregados. O PMM é uma variação na qual ao invés de um único valor de lucro, todo item possui $m$ valores ($m$ é o número de objetivos). O \ac{PMM} é uma abordagem teórica e apesar de testar os algoritmos em seus extremos, devido a seu enorme espaço de busca, nem sempre reflete a realidade dos problemas cotidianos. O \ac{PRM}, por sua vez, é um problema prático geralmente encontrado em comunicações de rede no qual se deseja transmitir uma mensagem de um dispositivo fonte para múltiplos destinos em uma rede de computadores, com o objetivo de utilizar os recursos disponíveis da forma mais eficiente possível. Esse problema é de extrema importância considerando-se o grande número de transmissões multimídia e aplicações em tempo real que se beneficiariam de algoritmos eficientes para cálculos de rota. De acordo com os experimentos realizados no capítulo \ref{chapter_experimentos}, o MACO/NDS mostrou-se ser um método interessante devido a sua rapidez no problema do roteamento multicast e eficácia no problema da mochila.

Em cada problema, os vários AEMOs são avaliados em diversas formulações de objetivos, variando-se de 2 a 6 objetivos, analisando-se a qualidade das soluções obtidas e discutindo as características presentes nos algoritmos que propiciam a obtenção de determinado resultado. 

Os \acp{ACO} são pouco explorados na literatura multiobjetivo quando comparados aos algoritmos genéticos. Este trabalho expande a coleção de métodos inspirados em colônias de formigas ao propor um novo \textit{framework} ACO para problemas multiobjetivos com muitas funções de otimização (\textit{many-objective}). Os ACOs são meta-heurísticas criadas especialmente para problemas discretos, o que pode representar uma grande vantagem em relação aos AGs, tanto em matéria de tempo como em qualidade das soluções. O MACO/NDS utiliza várias tabelas de feromônios para guiar a população de formigas, uma para cada combinação possível de objetivos. Cada grupo de formigas é associada a uma estrutura de feromônios diferente e, a cada passo do algoritmo, a atualiza de acordo com as novas soluções ótimas encontradas. Cada tabela de feromônio representa um subproblema e é responsável por guiar as soluções de acordo com um conjunto de objetivos diferentes, dessa forma, o algoritmo proposto (MACO/NDS) começa explorando subproblemas mais simples e manipula os mais complexos no decorrer do algoritmo. Para o PRM, os experimentos deste trabalho revelaram que o MACO/NDS, junto ao MOACS são os algoritmos mais rápidos e eficazes. Para o PRM, o MACO/NDS e MOEA/D-ACO conquistam as melhores soluções. Ou seja, o MACO/NDS é um método interessante para ambos os problemas, e dentre os ACOs testados é que apresentou maior estabilidade considerando ambos os problemas e suas diferentes instâncias.

\section{Objetivos}
Esta dissertação estende os trabalhos de \citeonline{Lafeta2016} e \citeonline{Bueno2010} sobre o problema do roteamento multicast, introduzindo um novo tipo de heurística bio-inspirada multiobjetivo (as colônias de formigas) e um novo problema discreto (o problema da mochila multiobjetivo). O principal objetivo deste trabalho é propor um novo algoritmo ACO para otimização em problemas \textit{many-objectives} e compará-lo com as principais estratégias presentes na literatura multiobjetivo tanto de AGs quanto de ACOs. Em linhas gerais, esta pesquisa almeja:
 
\begin{itemize}
	\item Propor um modelo para a construção de soluções em algoritmos baseados em colônias de formigas, de acordo com as características de cada um dos problemas investigados (PMM e PRM). Tais modelos são partes essenciais para a proposição do algoritmo ACO.
	\item Propor o novo algoritmo baseado em \ac{ACO} para problemas com muitos objetivos: O \ac{MACO/NDS}. 
	\item Investigar dois problemas multiobjetivos discretos. O problema do roteamento multicast (PRM) foi utilizado para comparar o desempenho de AEMOs nos trabalhos anteriores do nosso grupo de pesquisa. A fim de melhor entender a influência do tipo do problema no desempenho/eficiência dos algoritmos estudados, investigou-se outro problema neste trabalho, o \ac{PMM} que, apesar de ter uma aplicação mais restrita, apresenta diferenças interessantes em relação ao \ac{PRM}, possibilitando uma análise mais profunda sobre comportamento dos vários algoritmos.
	\item Para ambos os problemas (\ac{PMM} e \ac{PRM}), adaptar os algoritmos encontrados na literatura e analisar o comportamento de cada um em relação à complexidade do espaço de busca e ao número de objetivos a fim de guiar as decisões sobre a construção do algoritmo proposto MACO/NDS.
	\item Além das métricas dependentes do conhecimento sobre o conjunto Pareto ótimo do problema, empregar uma métrica não paramétrica. Neste caso, o hipervolume foi utilizado, ele permite avaliar o desempenho dos algoritmos multi e \textit{many-objective} em problemas nos quais a fronteira de Pareto não é conhecida. Essa métrica permitiu comprovar a vantagem dos ACOs (inclusive do método proposto, MACO/NDS) sobre os AGs em instâncias complexas do PMM e do PRM, onde não é possível estimar as fronteiras de Pareto.
\end{itemize}


\section{Contribuições}
Este trabalho contribui para o campo de otimização multiobjetivo, assim como o de comunicações em rede (através do problema do roteamento multicast). Os principais resultados desta pesquisa são resumidos nos seguintes tópicos:

\begin{itemize}  
	\item O \ac{MEAMT}, ou \ac{AEMMT}, em português, foi proposto originalmente para predição de estruturas de proteínas \cite{Brasil2013} e posteriormente, utilizado em  \cite{Lafeta2016} para resolver o problema do roteamento multicast. Neste trabalho, explora-se uma nova aplicação do algoritmo ao utilizá-lo para encontrar soluções para uma versão multiobjetivo do problema da mochila.
	\item O \ac{MEANDS}, ou \ac{AEMMD}, em português, foi proposto e analisado em \cite{Lafeta2016} para resolver o \ac{PRM}. Sua eficácia como algoritmo multiobjetivo em outro problema diferente do roteamento ainda não havia sido investigada. Neste trabalho, mostra-se como é possível aplicar o \ac{AEMMD} também ao \ac{PMM}.
	\item Através da execução de vários algoritmos multi e \textit{many-objective} sobre 2 problemas diferentes em diversos níveis de complexidade, foi feita uma análise sobre o comportamento desses algoritmos à medida em que se aumenta o número de objetivos e a complexidade do espaço de busca, possibilitando uma comparação entre os principais algoritmos da literatura.
	\item A principal contribuição desta dissertação está na proposição de um novo algoritmo denominado \textit{Many-objective Ant Colony Optimization based on Non-Dominated Sets} (MACO/NDS). O MACO/NDS é uma estratégia baseada em colônia de formigas e decomposição de objetivos que resolve problemas com muitos objetivos de forma rápida e eficiente. O novo método foi aplicado aos problemas \ac{PMM} e \ac{PRM} e comparado com os demais algoritmos. Durante o processo de desenvolvimento desse algoritmo, diferentes estratégias para a construção de soluções em ambos os problemas (PMM e PRM) foram avaliadas e diversos experimentos foram realizados, os quais serão apresentados no decorrer do texto.
\end{itemize}

\section{Organização do texto}
Este trabalho está dividido em capítulos organizados da seguinte forma:

\begin{itemize}  
	\item \textbf{Capítulo 2, otimização bio-inspirada}: nesse capítulo apresenta-se uma visão geral sobre os algoritmos genéticos e a otimização por colônia de formigas.
	\item \textbf{Capítulo 3, otimização multiobjetivo}: apresenta-se a definição de problemas multiobjetivos e descreve-se cada algoritmo usado neste trabalho.
	\item \textbf{Capítulo 4, problemas de teste}: introduz os dois problemas explorados nesta dissertação, o Problema da Mochila Multiobjetivo (PMM) e o Problema do Roteamento Multicast (PRM), assim como as estratégias evolutivas para cada um deles.
	\item \textbf{Capítulo 5, algoritmo proposto}: descreve o algoritmo \ac{MACO/NDS} proposto nesta dissertação e as estratégias para a construção de soluções em ACO para ambos os problemas tratados neste trabalho (PMM e PRM).
	\item \textbf{Capítulo 6, experimentos}: apresenta e discute todos os experimentos realizados no decorrer deste trabalho com o objetivo de avaliar a eficiência do algoritmo proposto, bem como analisar comparativamente o desempenho das abordagens investigadas na resolução de diferentes configurações de dois problemas discretos distintos.
	\item \textbf{Capítulo 7, conclusão}: resume as principais conclusões obtidas a partir dos experimentos e apresenta ideias para trabalhos futuros.
\end{itemize}