\chapter[Otimização bio-inspirada]{Otimização bio-inspirada}

Grande parte dos problemas de otimização envolvem encontrar a melhor opção num conjunto de possibilidades que cresce de maneira exponencial. Tais problemas são impossíveis de se resolver em tempo viável e necessitam de estratégias inteligentes para aproximar-se da solução ótima em tempo hábil. Dentre essas estratégias destacam-se os algoritmos gulosos [] e os algoritmos de busca bio-inspirados [].

Os algoritmos gulosos são algoritmos relativamente simples que, apesar de nem sempre obterem a solução ótima, normalmente aproximam-se bem dela [14]. Tais métodos são geralmente utilizados quando apenas um critério é envolvido na otimização, quando mais de um objetivo deve ser analisado, o problema se torna bem mais complexo e essa abordagem deixa de ser indicada.

A otimização bio-inspirada lança mão de estratégia baseadas na natureza para se encontrar boas soluções de forma eficiente. Assim como os algoritmos gulosos, não é possível garantir que se encontre exatamente a solução ótima, mas com uma boa modelagem do problema, é possível encontrar soluções suficientemente próximas. Na natureza, o processo de encontrar uma melhor solução aparece a todo momento, desde a forma como os primeiros animais surgiram, até a maneira como uma simples formiga encontra o caminho mais curto entre a colônia e uma fonte de comida. Ao observar processos comuns da natureza, estudiosos encontraram maneiras simples e eficientes de se resolver diversos problemas de otimização matemática [].

Os principais métodos bio-inspirados na literatura de otimização matemática são os algoritmos genéticos (AG), as colônias de formigas (ACO) e os enxames de partículas (PSO). Os algoritmos genéticos se inspiram na teoria da evolução de Darwin. Segundo [], cada indivíduo possui um material genético que é cruzado com os genes de outro representante da espécie para gerar um novo indivíduo, durante tal cruzamento ainda podem ocorrer mutações aleatórias que podem tanto gerar características benéficas quanto maléficas. O ambiente determina a parte da população que sobrevive e a parte que perece. Os indivíduos sobreviventes, ou seja, aqueles bem adaptados ao ambiente, possuem maiores chances de se reproduzir e espalhar suas boas características genéticas. Desta forma, o processo de evolução das espécies nada mais faz do que gerar indivíduos e selecionar os melhores, repetindo o processo até que se obtenha indivíduos bem adaptados ao ambiente em questão. 

Os ACOs são inspirados no comportamento das formigas, organismos simples que quando analisados em conjunto (colônia) apresentam comportamento complexo. O interesse nas formigas vem da observação de que ao buscarem por comida, acabam por encontrar o caminho mais rápido entre a fonte de alimento e o formigueiro. Como podem seres tão simples resolverem eficientemente um problema de otimização? Estudos revelaram que as formigas se baseiam no depósito de feromônios para se guiarem, quanto mais forte o feromônio em um caminho, maior a chance do mesmo ser tomado. Tal comportamento serviu de inspiração para os algoritmos de otimização bio-inspirados, dando origem ao ACO. Os PSOs, assim como os ACOs são inspirados no comportamento emergente de populações de animais simples. A otimização por enxame de partículas se inspira na navegação de pássaros, onde cada elemento da formação se guia através dos pássaros à frente. O algorítimo se baseia na direção e velocidade de cada elemento do enxame, determinando a exploração do espaço de busca através de operações vetoriais, portanto é uma estratégia indicada para problemas contínuos, que não é o caso dos problemas explorados nesta dissertação.

Em geral, todo algoritmo de otimização bio-inspirado inicia sua busca através da geração de soluções aleatórias e então entra em um laço, onde as melhores soluções guiam a construção de novas soluções que serão submetidas ao mesmo processo até que uma condição de parada seja atingida. Como não é necessário gerar todas as soluções possíveis, são métodos eficazes que, quando bem modelados, encontram um conjunto de boas soluções que resolvem o problema. Uma das principais diferenças entre os algoritmos bio-inspirados e as demais estratégias de otimização é o fato de que os primeiros produzem um conjunto de soluções aproximadas, o que oferece ao usuário uma gama de boas solução para que possa ele mesmo decidir qual utilizar.

\section{Algoritmos Genéticos (AGs)}
Os algoritmos genéticos são métodos de busca baseados na teoria da evolução de Charles Darwin [6]. A teoria de Darwin, hoje já endossada por diversas observações no campo da biologia, parte do princípio de que os organismos se adaptam ao ambiente em que vivem através de mutação, cruzamento e seleção natural. De maneira aleatória, um indivíduo em uma população pode ter alguma característica alterada, esse novo atributo pode ajudá-lo a sobreviver em seu habitat ou atrapalhá-lo. Os indivíduos com mutações favoráveis têm maiores chances de sobreviver e reproduzir, passando suas características benéficas para a geração seguinte. Dessa forma, ao longo de milhões de anos, organismos simples se tornam complexos e extremamente adaptados ao meio.

A ideia dos algoritmos genéticos é aplicar o mesmo conceito da evolução natural na computação, a ideia é partir de soluções aleatórias e após várias iterações de mutação, crossover e seleção encontrar um conjunto de soluções que resolvem bem o problema. Nesta ideia, o indivíduo na população representa uma solução, o meio representa o problema e as operações de mutação e crossover devem ser definidas, respectivamente, de forma a permitir uma alteração aleatória em uma solução e a combinação de duas soluções.

Em um algoritmo genético, primeiramente gera-se uma população inicial com indivíduos aleatórios e calcula-se a aptidão de cada um. No caso de um problema de otimização para uma função objetivo $f$, a aptidão de um indivíduo $x$ é dada por $f(x)$. Após esse processo de inicialização parte-se para a execução do laço do AG. O laço do AG consiste em:

\begin{enumerate}  
	\item Sortear os pares de pais; 
	\item Aplicar o crossover em cada par e gerar os filhos; 
	\item Aplicar a mutação aos filhos de acordo com uma taxa de mutação pré-estabelecida;
	\item Avaliar os filhos;
	\item Selecionar os melhores entre pais e filhos para formar a população da iteração seguinte (elitismo).
\end{enumerate}

O laço do AG termina quando uma condição de término estabelecida pelo usuário é atingida, e.g., 100 iterações.

\subsection{Representação do indivíduo}
A principal dificuldade ao se elaborar um algoritmo genético é definir a representação do indivíduo, cada um representa uma possível solução para o problema e deve ser codificado em uma estrutura que permita a mutação e a troca de características com outro indivíduo (cruzamento). Na proposição original do AG [], o indivíduo é representado de forma binária, ou seja, a solução para o problema é codificada em bits, que podem facilmente ser invertidos (mutação) ou copiados de um elemento para outro (cruzamento).

No problema da mochila 0/1, por exemplo, existe um conjunto de itens $I$ com pesos e valores e uma mochila com capacidade limitada. Deve-se descobrir qual a melhor forma de se arranjar os itens de maneira que o soma dos valores de cada um seja máxima e que a capacidade da mochila não seja excedida. Para representar uma solução deste problema em um AG, basta assumir um vetor binário de tamanho igual a $|I|$, onde diz-se que o item está na mochila se sua posição correspondente no vetor binário é 1 e não está, caso contrário.

Outras representações de indivíduos que não binárias também são possíveis, mas apresentam novos desafios. Em problemas de menores caminhos, por exemplo, normalmente trabalha-se com grafos. Logo, um indivíduo é um grafo e tanto a mutação quanto o cruzamento deverão ser operações em grafos.

para elaborar a representação do indivíduo no AG, deve-se levar em consideração a facilidade de manipulação da estrutura, a possibilidade de se introduzir um fator aleatório (mutação) e, principalmente, a representação das características de ambos os pais nos filhos, se a estrutura não permite a herança de características, não é uma boa escolha para se utilizar num algoritmo genético. Além disso, é preciso se certificar que cada solução pode ser representada de uma única maneira e que cada indivíduo pode representar uma única solução (relação um-para-um).

\subsection{Operadores Genéticos}
Nos algoritmos genéticos destacam-se três operações principais: cruzamento (ou crossover), mutação e seleção. O cruzamento e a mutação estão diretamente ligados com a escolha da representação do indivíduo, enquanto a seleção opera sobre a avaliação (fitness) de cada elemento da população.

