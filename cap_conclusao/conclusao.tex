\chapter[Conclusão]{Conclusão}
Este trabalho estendeu a pesquisa de \cite{Bueno2010} e \cite{Lafeta2016}, adicionando um novo problema de teste, o problema da mochila multiobjetivo (PMM), e 6 novos algoritmos \textit{many-objective}, sendo três evolutivos (MOEA/D, SPEA2-SDE e NSGA-III) e três baseados em colônias de formigas (MOACS, MOEA/D-ACO e MACO/NDS). Dentre os ACO's incluídos no trabalho, um deles, o \textit{Many-objective Ant Colony Optimization based on Non-Dominated Sets} (MACO/NDS), foi proposto pelo autor juntamente com uma nova estratégia de construção de solução para o problema do roteamento multicast (PRM).

O primeiro objetivo do trabalho, representado pela etapa 1 de experimentos (seção \ref{section_experimentos_etapa1}), foi incluir o novo problema PMM e comparar os algoritmos NSGA-II, SPEA2, MOEA/D, NSGA-III e AEMMT utilizando as métricas de desempenho paramétricas (baseadas no conhecimento prévio da fronteira de Pareto): taxa de erro ($ER$), \textit{generational distance} ($GD$) e \textit{pareto subset} ($PS$). Todos os métodos foram executados para ambos os problemas (PRM e PMM) em diversos cenários, variando a complexidade das instâncias e as quantidades de objetivo. O estudo permitiu verificar o comportamento de cada estratégia multiobjetivo em relação tanto à complexidade da entrada quanto ao número de funções de otimização. De maneira geral, confirmou-se a expectativa de se encontrar melhores resultados para os algoritmos clássicos NSGA-II e SPEA2 nos cenários com 2 e 3 objetivos e a decadência nos seus desempenhos a partir de 4 objetivos. Com 4 ou mais objetivos, os algoritmos MOEA/D e AEMMT obtiveram resultados bem melhores que os demais. O NSGA-III, por sua vez, se mostrou o método mais estável: não é o pior nem o melhor na maioria das situações. Para os problemas \textit{many-objective}, foco deste trabalho, o AEMMT foi o algoritmo que se mostrou mais interessante.

Os AG's \textit{multi} e \textit{many-objective} estão massivamente presentes na literatura e já foram explorados de diversas maneiras possíveis, por outro lado, os algoritmos baseados em colônias de formigas, apesar do potencial, não são utilizados com a mesma frequência. Os ACO's, por utilizarem uma estratégia de busca diferente, podem encontrar soluções até então inexploradas. Por esse motivo, esta pesquisa estudou vários ACO's na literatura, implementou dois deles e propôs um novo algoritmo, o MACO/NDS. Um dos principais aspectos de um ACO é a construção da solução, assim como num AG um dos principais fatores é o processo de \textit{crossover}. Dessa forma, este trabalho também traz uma revisão dos métodos para se construir uma solução no PMM (seção \ref{section_algoritmo_pmm}) e uma nova estratégia para se criar as árvores do PRM (seção \ref{section_algoritmo_prm}).

Considerando que os problemas com 2 e 3 objetivos são bem resolvidos pelos algoritmos multiobjetivos tradicionais (NSGA-II e SPEA2), mas que estes não são eficientes para lidar com problemas a partir de 4 objetivos, a partir da etapa 3 de experimentos, o foco desta pesquisa foi nos problemas e algoritmos \textit{many-objective}. Assim, nessa etapa foi analisado pela primeira vez o comportamento do MACO/NDS contra os algoritmos genéticos NSGA-III, MOEA/D, AEMMT e AEMMD nos três cenários mais simples de cada problema. A análise foi feita através das métricas baseadas em Pareto ($ER$, $GDp$ e $PS$) e do tempo de execução. Em geral, no PRM, o AEMMD e o método proposto, MACO/NDS, obtiveram os melhores resultados. O AEMMD consegue soluções um pouco melhores, mas ao mesmo tempo leva até quatro vezes mais tempo para executar. O MACO/NDS, apesar de não conseguir o melhor conjunto de soluções entre os dois métodos, chega em resultados bem próximos e é um algoritmo muito rápido, característica essencial para um problema de roteamento em redes. Com relação ao PMM, devido ao tamanho muito grande das fronteiras de Pareto, o MACO/NDS não exibe um comportamento tão bom em relação ao tempo, mas é consideravelmente o melhor algoritmo em relação ao $PS$, atingindo valores muito bons de $ER$ e $GDp$.

Tendo comprovado a eficácia do MACO/NDS em relação aos algoritmos genéticos utilizando as instâncias mais simples de cada problema, na última etapa de experimentos (seção \ref{section_experimentos_etapa4}), foram avaliadas instâncias mais complexas do PMM e do PRM. Além disso, para que fosse possível aferir a qualidade do MACO/NDS em relação a outras estratégias baseadas em colônias de formigas, incluiu-se os algoritmos MOEA/D-ACO e MOACS. Um novo algoritmo genético também foi incluído nas comparações, o SPEA2-SDE, uma variação do SPEA2 com a adição do operador SDE, tornando esse AEMO mais eficiente ao lidar com problemas que envolvam 4 ou mais objetivos. Como forma de avaliação dos resultados, utilizou-se o hiper-volume devido à dificuldade de se extrair um Pareto aproximado para as redes 4 e 5 e os problemas da mochila com 100 e 200 itens. No PRM, foi possível verificar que os ACO's levam normalmente menos tempo para executar que os AG's. Dentre os algoritmos genéticos, o único método com bons resultados foi o AEMMD (exclusivamente no PRM). O SPEA2-SDE consegue ótimas soluções, mas o alto custo do algoritmo em espaços de alta dimensionalidade faz com que ele seja uma opção inviável para o PRM. Em geral, os ACO's apresentaram uma melhor relação de custo/benefício no PRM e dentre eles, o MOACS obteve melhor hiper-volume e tempo na maioria dos casos, enquanto o MACO/NDS mostrou uma tendência de obter o melhor conjunto de soluções à medida que cresce a complexidade da entrada. Quanto ao PMM, os algoritmos se comportaram de maneira mais estável ao variar os cenários de testes. Para todos os casos do PMM, o MOEA/D-ACO apresentou o melhor custo benefício entre hiper-volume e tempo, sendo o MOEA/D uma alternativa a ser considerada quando é necessário um tempo de execução muito curto.

Desta forma, as principais contribuições deste trabalho para o campo de busca e otimização multiobjetivo foram:

\begin{itemize}
	\item Comparação entre AG's: foram comparados 7 algoritmos genéticos multiobjetivos em dois problemas discretos diferentes, o que oferece uma grande gama de dados para que possam ser tomadas decisões a respeito de qual algoritmo utilizar em determinadas situações.
	\item Proposição de um novo ACO e um modelo de construção de soluções para o PRM: este trabalho propôs uma nova ideia para se implementar ACOs multiobjetivos, apresentando um algoritmo eficaz que lida bem com problemas de muitos objetivos em um campo relativamente pouco explorado (otimização multiobjetivo por colônias de formigas). O novo algoritmo de construção da solução apresentado para o PRM, não só é parte da proposição do MACO/NDS, como também pode ser utilizado em outros \textit{frameworks} para ACO já estabelecidos, melhorando o desempenho desses algoritmos. Isso foi observado, por exemplo, ao se avaliar o método MOACS utilizando o novo modelo de construção e o modelo original proposto em \cite{Riveros2016} para o PRM, onde o novo modelo foi superior.
	\item Comparação entre AGs e ACOs: outra contribuição desta pesquisa foram as comparações de desempenho entre os algoritmos genéticos e as algoritmos baseados em colônias de formigas, considerando diferentes cenários para cada problema (PRM e PMM) e utilizando  métricas paramétricas (erro, $GD$ e $PS$) e não paramétricas (hiper-volume e tempo de execução). Por meio das análises foi possível identificar alguns pontos fortes e fragilidades de cada metaheurística.
\end{itemize}

\section{Trabalhos Futuros}
Algumas novas linhas de investigação foram vislumbradas no decorrer deste trabalho e seria muito interessante abordá-las no futuro. Propomos a investigação dos seguintes tópicos em pesquisas futuras:

\begin{itemize}
	\item Em problemas onde se conhece a fronteira de Pareto, a métrica de avaliação \textit{inverse generation distance} (IGD) vem se tornando uma importante referência de avaliação nos trabalhos mais recentes sobre otimização multiobjetivo, até mesmo substituindo em alguns casos as métricas mais tradicionais $ER$, $GD$, $GDp$ e $PS$. Assim, um aspecto interessante na continuidade desse trabalho seria incluir essa métrica de avaliação nas análises futuras. %Outra métrica interessante para pesquisas futuras que, por outro lado, não precisa do conhecimento prévio sobre o Pareto é o \textit{HV-}, baseado na métrica de hiper-volume.
	
	\item Foi observado que o tempo de execução do MACO/NDS no PMM é muito alto, diferentemente do observado para o PRM. Tal comportamento se deve ao fato de o número de soluções não dominadas nas instâncias do PMM aqui investigadas ser muito elevado, enquanto que no PRM, a cardinalidade do Pareto não é tão significativa. Essa cardinalidade elevada no PMM, faz com que o algoritmo despenda um tempo significativo na manipulação das soluções não dominadas já encontradas. A investigação de alguma estratégia de limitação no tamanho do conjunto corrente de soluções não dominadas, que elimine soluções de forma eficiente é importante para reduzir o tempo de execução do algoritmo sem afetar muito a qualidade das soluções em relação ao Pareto real. Nessa investigação, estratégias usuais de métodos evolutivos podem ser avaliadas (ex: truncamento, clusterização, pontos de referência, etc).
	
	\item Investigar outros testes de hipótese além do Z-teste como forma de avaliação dos resultados obtidos por cada algoritmo, principalmente testes não paramétricos que permitem a comparação de mais de duas abordagens.

	\item O MACO/NDS e o MOACS executam muito mais rápido que o AEMMD em alguns cenários do PRM, mas produzem um conjunto de soluções com qualidade inferior. Em nossos experimentos, o número de avaliações de novos indivíduos foi utilizado como parâmetro para determinar uma comparação equilibrada entre os indivíduos. Por outro lado, os protocolos de internet exigem que na prática o roteamento seja realizado em um tempo bastante restrito (na ordem de grandeza de milisegundos). Seria interessante investigar o desempenho dos três algoritmos quando fossem executados sujeitos a uma restrição de tempo, independentemente do número de avaliações efetuadas de novos indivíduos.
	
	\item Nos cenários mais complexos do PRM, o MACO/NDS apresentou resultados significativamente melhores que o MOACS. Uma nova investigação utilizando redes ainda mais complexas (maior número de nós e arestas) e um número maior de objetivos seria interessante para verificar se essa tendência continuaria nos problemas mais difíceis de resolver, bem como tentar identificar se ela tem maior relação com a complexidade da rede ou ao número maior de objetivos.
	
	\item Em relação ao problema de roteamento multicast, foram investigadas algumas combinações de objetivos relacionados ao roteamento com qualidade de serviço, apresentados na seção \ref{section_prm_definicao}. Essas combinações (2 a 6 objetivos) foram definidas em função das formulações utilizadas em trabalhos anteriores - \citeonline{Lafeta2016, Bueno2010}. Como continuidade dessa investigação, formulações adicionais podem ser avaliadas, realizando-se uma análise de correlação entre os objetivos para a definição das novas combinações.

	\item O problema da mochila multiobjetivo investigado inclui múltiplos objetivos, mas apenas uma restrição. Essa característica faz com que o crescimento do número de soluções não-dominadas seja muito significativo quando o número de itens a serem alocados é aumentado. E esse crescimento é ainda intensificado quando um número maior de objetivos é utilizado. Por exemplo, pela Tabela \ref{table_exp3_paretos}, observa-se  que o Pareto aproximado no PMM com 6 objetivos subiu de 6491 para 55471, quando o número de itens subiu de 40 para 50. Essa característica tornou a aproximação do Pareto inviável para 100 e 200 de objetos. Por outro lado, existem outras variações do PMM que trabalham com múltiplos objetivos e múltiplas restrições \cite{Ishibuchi2015,Alaya2007}. Acredita-se que nessa variação do PMM, o crescimento do número de soluções não dominadas com o aumento do número de itens e/ou objetivos seja mais controlada. Seria interessante uma nova implementação utilizando-se essa variação do PMM, para verificar se o comportamento dos algoritmos mudaria muito. Além disso, com uma menor quantidade de elementos no Pareto, onde os algoritmos levariam muito menos tempo para executar, seria interessante avaliar se haveria alguma modificação na análise comparativa dos tempos de execução entre os métodos.
	
	\item O MACO/NDS foi avaliado em apenas dois problemas \textit{many-objective} nessa dissertação: PMM e PRM. Para validá-lo como um \textit{framework} mais geral, seria relevante a sua aplicação em outros problemas de otimização discreta com muitos objetivos. Como problemas em potencial para essa investigação futura, podemos destacar as funções discretas apresentadas em \cite{DiscreteFunctions1} e  \cite{DiscreteFunctions2}, o problema do caixeiro viajante \cite{Riveros2016} e o problema de escalonamento \textit{job shop} \cite{JobShop}, que também são investigados na literatura de otimização discreta com muitos objetivos.
	
	\item Avaliar a influência dos feromônios e da heurística nos resultados a fim de identificar as características dos problemas que se beneficiam do uso de um ou de outro. Estudar possíveis dinamizações dos parâmetros $\alpha$ e $\beta$ para que possam ajustar a influência dos dois componentes (feromônio e heurística) da melhor forma possível no decorrer do algoritmo.
	
	\item Embora existam adaptações de \textit{frameworks} ACOs para domínios contínuos, entendemos que a maior aplicabilidade do MACO/NDS seja em problemas de otimização combinatória, pela maior adaptabilidade da construção iterativa de soluções a problemas discretos. Entretanto, existe um outro método baseado em inteligência coletiva mais adequado a problemas contínuos que também é bastante investigado na literatura de otimização multiobjetivo: o PSO. Assim, um possível desdobramento da nossa pesquisa é a adaptação das ideias dos algoritmos baseados na decomposição em subproblemas multiobjetivos (MACO/NDS e AEMMD) para um \textit{framework} baseado em PSO e sua posterior aplicação a funções e problemas contínuos, além de compará-lo a outras abordagens similares \cite{Freire2017}.
\end{itemize}

\section{Contribuições em Produção Bibliográfica}
A produção bibliográfica originada neste trabalho compreende os seguintes artigos:

\begin{enumerate}
	\item ``\textit{A Comparative Analysis of MOEAs Considering Two Discrete Optimization Problems}'' por Tiago Peres França, Thiago Fialho de Q. Lafetá, Luiz G. A. Martins e Gina M. B. de Oliveira. Publicado e apresentado no evento \textit{Brazilian Conference on Intelligent Systems} (BRACIS) de 2017. Este artigo compreende os experimentos realizados na etapa 1 (seção \ref{section_experimentos_etapa1}). \cite{Franca2017}.
	\item ``\textit{MACO/NDS: Many-objective Ant Colony Optimization based on Decomposed
		Pheromone}'' por Tiago Peres França, Luiz G. A. Martins e Gina M. B. de Oliveira. Publicado no evento \textit{Congress on Evolutionary Computation} (CEC) de 2018. Este artigo propõe o algoritmo MACO/NDS (capítulo \ref{chapter_macod}) e compreende os experimentos realizados na etapa 3 (seção \ref{section_experimentos_etapa3}). \cite{Franca2018}.
\end{enumerate}