\chapter[Conclusão]{Conclusão}
Este trabalho estendeu a pesquisa de \cite{Bueno2010} e \cite{Lafeta2016}, adicionando um novo problema de teste, o problema da mochila multiobjetivo (PMM), e 6 novos algoritmos many-objectives, sendo 3 deles AG's (MOEA/D, SPEA2-SDE e NSGA-III) e 3 deles ACO's (MOACS, MOEA/D-ACO e MACO/NDS). Dentre os ACO's incluídos no trabalho, um deles, o \textit{Many-objective Ant Colony Optimization based on Non-Dominated Sets} (MACO/NDS), foi proposto pelo autor juntamente com uma nova estratégia de construção de solução para o problema do roteamento multicast (PRM).

O primeiro objetivo do trabalho, representado pela etapa 1 de experimentos (seção \ref{section_experimentos_etapa1}), foi incluir o novo problema PMM e comparar os algoritmos NSGA-II, SPEA2, MOEA/D, NSGA-III e AEMMT utilizando as métricas baseadas em Pareto: taxa de erro ($ER$), \textit{generational distanc}e ($GD$) e \textit{pareto subset} ($PS$). Todos os métodos foram executados para ambos os problemas (PRM e PMM) em diversos cenários variando a complexidade das instâncias e as quantidades de objetivo. O estudo permitiu verificar o comportamento de cada estratégia multiobjetivo em relação tanto à complexidade da entrada quanto ao número de funções de otimização. De maneira geral, confirmou-se a expectativa de se encontrar melhores resultados para os algoritmos clássicos NSGA-II e SPEA2 nos cenários com 2 e 3 objetivos e a decadência em performance dos mesmos a partir de 4 objetivos. Com 4 ou mais objetivos, os algoritmos MOEA/D e AEMMT obtiveram resultados bem melhores que os demais. O NSGA-III, por sua vez, se mostrou o método mais estável: não é o pior nem o melhor na maioria das situações. Para os problemas \textit{many-objectives}, foco deste trabalho, o AEMMT foi o algoritmo que se mostrou mais interessante.

Os AG's \textit{multi} e \textit{many-objectives} estão massivamente presentes na literatura e já foram explorados de diversas maneiras possíveis, por outro lado, os algoritmos baseados em colônias de formigas, apesar do potencial, não são utilizados com a mesma frequência. Os ACO's, por utilizarem uma estratégia de busca diferente, podem encontrar soluções até então inexploradas. Por esse motivo, esta pesquisa estudou vários ACO's na literatura, implementou dois deles e propôs um novo algoritmo, o MACO/NDS. Um dos principais aspectos de um ACO é a construção da solução, assim como num AG um dos principais fatores é o processo de \textit{crossover}. Dessa forma, este trabalho também traz uma revisão dos métodos para se construir uma solução no PMM (seção \ref{section_estrategias_pmm_aco}) e uma nova estratégia para se criar as árvores do PRM (seção \ref{section_estrategias_prm_aco}). Afim de se desenvolver o algoritmo proposto para a construção da solução no PRM, efetuou-se a segunda etapa de experimentos (seção \ref{section_experimentos_etapa2}) que comprovou a superioridade do modelo proposto.

Como os problemas multi-objetivos com 2 e 3 funções de otimização são considerados bem resolvidos e através da etapa 1 de experimentos comprovou-se a ineficácia dos algoritmos NSGA-II e SPEA2 para problemas com 4 ou mais objetivos, a partir da etapa 3 de experimentos descartam-se os cenários com 2 e 3 objetivos e deixa-se de incluir os métodos clássicos NSGA-II e SPEA2. Assim, na terceira etapa de experimentos, analisa-se pela primeira vez o comportamento do MACO/NDS contra os algoritmos genéticos NSGA-III, MOEA/D, AEMMT e AEMMD nos três cenários mais simples de cada problema. A análise é feita através das métricas baseadas em Pareto ($ER$, $GD$ e $PS$) e do tempo de execução. Em geral, no PRM, o AG AEMMD e o método proposto, MACO/NDS, obtiveram os melhores resultados. O AEMMD consegue soluções de qualidade um pouco melhor que seu oponente, mas ao mesmo tempo leva até quatro vezes mais tempo para executar. O MACO/NDS, apesar de não conseguir o melhor conjunto de soluções entre os dois métodos, chega bem próximo e é um algoritmo muito rápido, característica essencial para um problema de roteamento em redes. Com relação ao problema da mochila, devido ao tamanho muito grande das fronteiras de Pareto, o MACO/NDS não exibe um comportamento tão bom em relação ao tempo, mas ao mesmo tempo, é de longe o melhor algoritmo em respeito ao $PS$, atingindo valores muito bons de $ER$ e $GD$.

Tendo comprovado a eficácia do MACO/NDS em relação aos algoritmos genéticos utilizando as instâncias mais simples de cada problema, na última etapa de experimentos (seção \ref{section_experimentos_etapa4}), concentra-se apenas nas 3 instância mais complexas do PMM e do PRM, além disso, para que seja possível aferir a qualidade do MACO/NDS em relação a outras estratégias baseadas em colônias de formigas, incluiu-se os algoritmos MOEA/D-ACO e MOACS. Um novo algoritmo genético também foi incluído nas comparações, o SPEA2. Como forma de avaliação dos resultados, utilizou-se o hiper-volume devido à dificuldade de se extrair um Pareto aproximado para as redes 4 e 5 e os problemas da mochila com 100 e 200 itens. No PRM, foi possível verificar que todos ACO's, normalmente, levam menos tempo para executar que os AG's. Dentre os algoritmos genéticos, o único método com bons resultados foi o AEMMD. O SPEA2 consegue ótimas soluções, mas o alto custo do algoritmo em espaços de alta dimensionalidade faz com que ele seja uma opção inviável para o PRM. Em geral, os ACO's apresentaram uma melhor relação de custo/benefício e dentre eles, o MOACS obteve melhor hiper-volume e tempo na maioria dos casos enquanto o MACO/NDS mostrou uma tendência de obter o melhor conjunto de soluções à medida que cresce a complexidade da entrada. Quanto ao problema da mochila, os algoritmos se comportaram de maneira mais estável ao variar os cenários de testes. Para todos os casos do PMM, o MOEA/D-ACO apresentou o melhor custo benefício entre hiper-volume e tempo, fazendo com que o único outro algoritmo considerável seja o MOEA/D, quando realmente é necessário um tempo de execução muito curto.

Desta forma, as principais contribuições deste trabalho para o campo de busca e otimização multi-objetivo foram:

\begin{itemize}
	\item Comparação entre AG's: foram comparados 7 algoritmos genéticos multi-objetivos em dois problemas discretos diferentes, o que oferece uma grande gama de dados para que se possa tomar decisões a respeito de qual algoritmo utilizar em determinadas situações.
	\item Proposição de um novo ACO e modelo para o PRM: este trabalho propôs uma nova ideia para se implementar ACO's multiobjetivos, o que contribuiu para o meio acadêmico apresentando um algoritmo eficaz que lida bem com problemas de muitos objetivos em um campo relativamente pouco explorado (colônias de formigas multiobjetivo). O novo algoritmo de construção da solução apresentado para o PRM, não só é parte da proposição do MACO/NDS, como também pode ser utilizado em outros \textit{frameworks} ACO já estabelecidos, melhorando o desempenho do algoritmo.
	\item Comparação entre AG's e ACO's: outra grande contribuição desta pesquisa foram as comparações feitas entre os algoritmos genéticos e as colônias de formigas possibilitadas pelos experimentos das etapas 3 e 4 (seções \ref{section_experimentos_etapa3} e \ref{section_experimentos_etapa4}).
\end{itemize}

\section{Trabalhos Futuros}
Algumas duvidas surgiram no decorrer deste trabalho e seria muito interessante abordá-las no futuro. Em próximas pesquisas pretende-se investigas os seguintes tópicos:

\begin{itemize}
	\item Nos cenários onde se conhece a fronteira de Pareto, seria interessante utilizar a métrica de avaliação \textit{inverse generation distance} (IGD), mais comum que o $ER$, $GD$ e $PS$ nos trabalhos mais recentes sobre otimização multiobjetivo.
	\item O tempo de execução do MACO/NDS no PMM é muito alto, talvez seja possível criar algum tipo de limitação no tamanho do Pareto que elimine soluções de forma eficiente, diminuindo o tempo sem afetar muito a qualidade das soluções.
	\item O MACO/NDS e o MOACS executam muito mais rápido que o AEMMD em alguns cenários, mas produzem um conjunto de soluções com qualidade levemente inferior. Qual dos três algoritmos obteria o melhor resultado se fossem executados todos durante o mesmo espaço de tempo?
	\item Nos cenários mais complexos do PRM, o MACO/NDS apresenta resultados significativamente melhores que o MOACS, essa tendência continuaria ao testar instâncias de redes mais complexas? E quanto a um maior número de objetivos?
	\item O problema da mochila multiobjetivo aqui testado inclui múltiplos objetivos, mas apenas uma restrição. Algumas variações do PMM trabalham com múltiplos objetivos e múltiplas restrições. O comportamento dos algoritmos mudaria muito com essa outra abordagem? Com uma menor quantidade de elementos no Pareto, os algoritmos levariam muito menos tempo para executar?
	\item Foram testados 3 ACO's neste trabalho, como pesquisa futura seria interessante investigar outras ideias de ACO's adaptando-nas aos problemas da mochila e do roteamento.
	\item O MACO/NDS foi até agora testado em apenas dois problemas, para validá-lo como \textit{framework}, seria de extrema importância uma pesquisa envolvendo sua implementação em outros problemas discretos.
\end{itemize}

\section{Contribuições em Produção Bibliográfica}
A produção bibliográfica originada neste trabalho compreende os seguintes artigos:

\begin{enumerate}
	\item ``\textit{A Comparative Analysis of MOEAs Considering Two Discrete Optimization Problems}'' por Tiago Peres França, Thiago Fialho de Q. Lafetá, Luiz G. A. Martins e Gina M. B. de Oliveira. Publicado e apresentado no evento \textit{Brazilian Conference on Intelligent Systems} (BRACIS) de 2017. Este artigo compreende os experimentos realizados na etapa 1 (seção \ref{section_experimentos_etapa1}). \cite{Franca2017}.
	\item ``\textit{MACO/NDS: Many-objective Ant Colony Optimization based on Decomposed
		Pheromone}'' por Tiago Peres França, Luiz G. A. Martins e Gina M. B. de Oliveira. Aprovado no evento \textit{Congress on Evolutionary Computation} (CEC) de 2018 e pendente de publicação. Este artigo propõe o algoritmo MACO/NDS (capítulo \ref{chapter_macod}) e compreende os experimentos realizados na etapa 3 (seção \ref{section_experimentos_etapa3}).
\end{enumerate}