\chapter[Algoritmo proposto]{Algoritmo proposto}
\label{chapter_macod}

O algoritmo proposto neste trabalho é baseado em colônia de formigas (ACO) e foi chamado de \textit{Many-objective Ant Colony Optimization based on Decomposed Pheromone} (MACO/D), em português, otimização em colônia de formigas para muitos objetivos baseada em decomposição de feromônios \cite{Franca2018}. A proposição envolve os seguintes módulos:

\begin{itemize}
	\item \textit{Framework} ACO: algoritmo geral desenvolvido para se trabalhar com qualquer problema discreto.
	\item Modelo PMM: construção da solução para o problema da mochila multiobjetivo, apresentado na seção \ref{section_estrategias_pmm_aco}.
	\item Modelo PRM: construção da solução para o problema do roteamento multicast, apresentado na seção \ref{section_estrategias_prm_aco}.
\end{itemize}

Este capítulo apresenta o módulo principal do algoritmo, o \textit{framework}. framework. Nela são discutidos alguns conceitos gerais sobre a implementação multiobjetivo do ACO, bem como uma descrição mais detalhada das etapas de construção das soluções e atualização dos feromônios.

\section{Visão geral do algoritmo}

O MACO/D se baseia na ideia do AEMMD de se criar tabelas de dominância para diferentes combinações possíveis de objetivos, adaptando-na para um modelo de colônia de formigas.

A multiplicidade de objetivos impõe que algumas mudanças sejam feitas na ideia original do ACO. Isso envolve responder as seguintes perguntas:

\begin{itemize}
	\item Como representar os diferentes objetivos no depósito de feromônio?
	\item Como compor as múltiplas heurísticas em uma única função?
\end{itemize}

De acordo com \cite{Alaya2007}, essas questões podem ser resolvidas através de uma das seguintes opções:

\begin{itemize}
	\item \textbf{Várias colônias e múltiplos feromônios $(m+1, m)$:} considerando $m$ o número de objetivos, esse modelo utiliza $m + 1$ colônias de formigas e $m$ estruturas de feromônios. Cada colônia é associada a um único objetivo e otimiza apenas esse objetivo através de sua própria estrutura de feromônios. Uma colônia adicional, que representa o conjunto de todos os objetivos, utiliza os feromônios das outras colônias de forma aleatória. Ao criar uma solução, sorteia-se alguma das estruturas de feromônios para se utilizar a cada passo. Outra proposta, no modelo $(m+1, m)$ é utilizar a soma de todas as estruturas de feromônios ao criar uma solução na colônia extra. Quanto as heurísticas, cada colônia utiliza a função referente a seu objetivo, enquanto a colônia extra utiliza o somatório de todas as funções de heurística.
	\item \textbf{Colônia e feromônio únicos $(1,1)$:} este modelo utiliza apenas uma colônia de formigas e uma estrutura de feromônios. Assim como no modelo anterior, as heurísticas são somadas para montar uma solução. A única estrutura de feromônios representa todos os objetivos. A atualização dos feromônios se dá através de um arquivo que mantém todas as soluções não-dominadas encontradas. Toda solução não-dominada contribui com a mesma quantidade de feromônios, já que, de fato, são indiferenciáveis em termos de qualidade.
	\item \textbf{Única colônia e vários feromônios $(1,m)$:} considerando $m$ o número de objetivos, esse modelo utiliza uma colônia de formigas e $m$ estruturas de feromônios. Assim como nos demais modelos, as heurísticas são somadas para montar uma solução. A cada passo da construção da solução, sorteia-se aleatoriamente a estrutura de feromônios a se utilizar. Cada estrutura de feromônios representa um objetivo e sua atualização é feita a cada iteração do algoritmo, de acordo com a melhor solução encontrada considerando aquele objetivo.
\end{itemize}

Em \cite{Alaya2007}, os experimentos mostraram que o terceiro modelo ($1,m$) gera os melhores resultados quando aplicado ao problema da mochila multiobjetivo. Dentre os ACOs multiobjetivos investigados, o MOACS se encaixa na segunda abordagem (1,1), adaptando apenas o modo de lidar com a heurística, enquanto o MOEA/D-ACO adota a primeira estratégia (m,m), pois trabalha com múltiplas formigas e várias estruturas de feromônios, apesar da quantidade exata de formigas e tabelas de feromônios não ser ditada pelo número de objetivos.

Nosso algoritmo (MACO/D), não pertence a nenhuma das categorias apresentadas por \cite{Alaya2007}, mas se aproxima da estratégia (1,m), pois lida com uma única colônia e múltiplos feromônios. A grande diferença é que o número de estruturas de feromônio não é igual a quantidade de objetivos, mas ao número de possíveis combinações entre os objetivos, de forma semelhante ao que ocorre no AEMMD \cite{Lafeta2017}. Além disso, o processo de cálculo da heurística e de atualização dos feromônios se difere substancialmente da ideia original do modelo $(1,m)$. O MACO/D, se baseia na ideia de decomposição, ou seja, ao invés de trabalhar diretamente com todos os objetivos, o problema é atacado em várias frentes com quantidades reduzidas de funções. Desta forma, evita-se o problema onde não é possível classificar a população devido ao alto número de soluções não-dominadas em espaços de dimensionalidade alta \cite{Deb2014}.

O primeiro passo do MACO/D é criar as estruturas de feromônio $P = \{p_1, p_2, ..., p_i, ..., p_n\}$, uma para cada combinação possível de objetivos (análogo ao processo do AEMMD apresentado na seção \ref{section_aemmd}). Para um problema de seis objetivos, por exemplo, são criadas 57 estruturas ($|P| = 57$). Cada $p_i \in P$ é responsável por um subproblema e guarda as seguintes informações:

\begin{itemize}
	\item \textbf{Valores:} corresponde aos valores dos feromônios em si, os quais são inicializados com o menor valor possível.
	\item \textbf{Objetivos:} determina os objetivos do subproblema em questão. É um vetor binário de tamanho $m$ (número de objetivos), onde cada o bit 1 representa um objetivo que faz parte do problema e o bit 0 indica aqueles que não pertencem ao problema.
	\item \textbf{Arquivo:} conjunto de soluções não-dominadas que apareceram até o momento para o subproblema em questão.
	\item \textbf{Convergência:} indica a convergência do arquivo. Começa em 0 e incrementa em 1 sempre que o arquivo sofre alterações durante uma iteração (época).
	\item \textbf{$\beta$:} importância da heurística no cálculo de probabilidades ao construir uma solução. Sempre inicializado com o valor do parâmetro $\beta$ passado ao MACO/D.
\end{itemize}

As estruturas de feromônios são utilizadas no processo de construção da solução e devem ser atualizadas em toda iteração do algoritmo. Manipular todas as estruturas de feromônios de $P$ (57 estruturas, no caso de seis objetivos) é computacionalmente caro e inviável. Por essa razão, define-se um número máximo de estruturas para se trabalhar em um dado momento. Em ambos os problemas utilizados neste trabalho (PMM e PRM), foi utilizado um limite de cinco estruturas de feromônios ativas simultaneamente. A ordem de ativação dos elementos de $P$ é determinada pela sua ordem em $P$. Por isso, é importante a sequência em que as estruturas são criadas: os primeiros subproblemas instanciados são aqueles que representam um menor número de  objetivos, ou seja, combinações 2 a 2. Em seguida, instancia-se os subproblemas de 3 objetivos e, assim por diante, até que a última estrutura de feromônios com o número total de objetivos seja criada.

Inicialmente, os cinco subproblemas mais simples são ativados ($P_{ativo} = \{p_1, p_2, ..., p_5\}$). Assim que algum deles passa a não contribuir satisfatoriamente para o conjunto de soluções, é desativado em favor do próximo problema mais complexo ainda não utilizado. Dessa forma, o conjunto de soluções do MACO/D cresce gradualmente, partindo das decomposições mais simples em direção às mais complexas. O desafio principal em uma otimização com muitos objetivos é diferenciar soluções não dominadas e é nesse ponto que se utiliza o auxílio da decomposição. Em cada iteração do laço principal, após gerar o conjunto de soluções $S$, extrai-se as soluções não-dominadas de acordo com todos os objetivos e, a partir desse subconjunto ($S_{nd}$), alimenta-se as cinco estruturas ativas em $P_{ativo}$. A distribuição das soluções não-dominadas $S_{nd}$ entre as estruturas ativas se dá de acordo com o Algoritmo \ref{alg_macod_dist}.

\begin{algorithm}
	\caption{Distribuição de soluções não dominadas entre as estruturas de feromônios}
	\label{alg_macod_dist}
	\begin{algorithmic}[1]
		\For{$a \in P_{ativo}$}
			\State atualize o arquivo de $a$ com as soluções em $S_{nd}$
			\State crie um conjunto $A$ com as soluções de $S_{nd}$ que foram adicionadas ao arquivo
			\State crie um conjunto $R$ com as soluções do arquivo que foram removidas
			\If{|A| > 0}
				\State defina o valor $\beta$ de $a$ para o valor padrão defino pelo parâmetro $\beta$ do algoritmo
				\State incremente os valores de feromônio de $a$ de acordo com as soluções em $A$
				\State decremente os valores de feromônio de $a$ de acordo com as soluções em $R$
			\Else
				\State incremente a convergência de $a$ em 1
				\State diminua a importância das heurísticas alterando o valor $\beta$ de $a$ através de um parâmetro pré-definido
				\If{convergência de $a$ atingiu o valor máximo}
					\State em $P_{ativo}$, substitua a estrutura $a$ pelo próximo $p_i \in P$
					\State reinicie a convergência de $a$
				\EndIf
			\EndIf
		\EndFor
	\end{algorithmic}
\end{algorithm}

O laço principal do ACO, que consiste em gerar soluções e atualizar as estruturas de feromônios ativas, termina quando o número máximo de iterações é atingido. Se todo $p \in P$ se torna ativo antes do término do programa, volta-se a explorar o início da lista, de maneira circular.  O valor máximo de convergência para cada estrutura utilizado neste trabalho foi 10. Ao final, o algoritmo retorna como solução o arquivo de $p_n$, ou seja, todas as soluções não-dominadas encontradas para a estrutura que considera todos os $m$ objetivos. O Algoritmo \ref{alg_macod_geral} apresenta uma visão geral do MACO/D.

\begin{algorithm}
	\caption{Algoritmo geral do MACO/D}
	\label{alg_macod_geral}
	\begin{algorithmic}[1]
		\State crie as estruturas de feromônios $P = \{p_1, p_2, ..., p_i, ..., p_n\}$ em ordem crescente de número de objetivos
		\State $P_{ativo} \gets \{p_1, p_2, ..., p_5\}$
		\While{número máximo de iterações não for atingido}
			\State construa o conjunto de soluções $S$ de acordo com os feromônios em $P_{ativo}$
			\State atualize o arquivo de $p_n$ com as soluções em $S$, chame-o de $S_{nd}$
			\State distribua as soluções em $S_{nd}$ entre as estruturas em $P_{ativo}$ (algoritmo \ref{alg_macod_dist})
		\EndWhile
		\State \Return arquivo de $p_n$
	\end{algorithmic}
\end{algorithm}

No Algoritmo \ref{alg_macod_dist}, as linhas 6 e 11 trabalham com a manipulação do valor $\beta$ da estrutura de feromônios. A intenção é que se diminua a importância das heurísticas sempre que a busca parecer estagnada. Assim que novas soluções são encontradas, o valor de $\beta$ volta ao normal. No caso da implementação utilizada neste trabalho, para diminuir a importância da heurística, deve-se aumentar o valor de $\beta$. A quantidade em que se deve alterar esse valor dependerá do problema. Nos nossos experimentos (descritos no capítulo \ref{chapter_experimentos}), o valor de $\beta$ aumenta em 10\% tanto no PMM quanto no PRM. Ainda no Algoritmo \ref{alg_macod_dist}, as linhas 7 e 8 fazem parte do processo de atualização de feromônios, o qual será melhor explicado na Seção \ref{section_macod_pheromones}. No algoritmo \ref{alg_macod_geral}, a linha 4 representa a construção das soluções, tema da seção seguinte.

\section{Construção das soluções}

As soluções em colônias de formigas são construídas a partir do grupo de feromônios ativo ($P_{ativo}$), das heurísticas ($H$) e dos valores de $\alpha$ e $\beta$. Os feromônios são criados e atualizados no decorrer do algoritmo, enquanto os demais são parâmetros de entrada. A heurística é uma função que estima a qualidade de uma parcela da solução, ex: arestas em problemas que envolvam grafos, como o PRM, e itens em problemas representados por vetores, como o PMM. Enquanto o parâmetro $\alpha$ determina a importância do feromônio ao tomar uma decisão a respeito da composição da solução, $\beta$ representa a importância da heurística.

No MACO/D existem múltiplas estruturas de feromônios e heurísticas. Portanto, para que se possa construir uma solução, é necessário antes escolher quais valores de feromônio e qual função de heurística serão utilizados. 

As estruturas de feromônios são recebidas pelo processo de construção das soluções através da lista de estruturas ativas do MACO/D ($P_{ativo}$), ou seja, 5 conjuntos de feromônios são utilizados por vez. De maneira circular, cada solução utiliza uma única estrutura de feromônios para ser construída, ou seja, a primeira solução é criada a partir da primeira estrutura de $P_{ativo}$, a segunda a partir da segunda e, assim por diante, até a sexta solução que utiliza novamente a primeira estrutura de $P_{ativo}$.

Com relação às heurísticas, o MACO/D utiliza o mesmo processo proposto em \cite{Riveros2016}. Admite-se um grau de importância para cada função: 0 (não importante), 1 (importante) ou 2 (muito importante). O valor de importância é sorteado para cada heurística e funciona como um peso. A função única utilizada para construir a solução é a soma ponderada das heurísticas considerando os pesos sorteados.

O processo geral para se construir as soluções é exposto no Algoritmo \ref{alg_macod_solucao}.

\begin{algorithm}
	\caption{Construção das soluções}
	\label{alg_macod_solucao}
	\begin{algorithmic}[1]
		\State $S \gets \emptyset$
		\For{$i \gets 0$ até nº máximo de soluções}
			\State sorteie valores entre 0 e 2 (inclusive) para um vetor $W$ de $|H|$ posições
			\State defina $h$ como a função a seguir: $h(x) = \sum_{i \gets 0}^{|H|-1}\frac{H[i](x) * W[i]}{\sum\limits_{w \in W}w}$
			\State chame de $f$ os valores de feromônio da estrutura $P_{ativo}[i \pmod{|P_{ativo}|}]$
			\State gere uma solução $s$ de acordo com os valores de feromônio $f$, a heurística $h$, $\alpha$ e $\beta$
			\State $S \gets S \cup \{s\}$
		\EndFor
		\State \Return $S$
	\end{algorithmic}
\end{algorithm}

Na linha 6 do algoritmo \ref{alg_macod_solucao} constrói-se a solução em si. Esse processo depende exclusivamente do problema em questão e representa a principal parte na elaboração do modelo para um algoritmo baseado em colônia de formigas. No caso do problema da mochila multiobjetivo, a estratégia utilizada é aquela descrita em \ref{section_estrategias_pmm_aco}. Para o problema do roteamento multicast, a estratégia de construção da solução foi apresentada em \ref{section_estrategias_prm_aco}.

\section{Atualização dos feromônios}
\label{section_macod_pheromones}

A atualização dos feromônios no MACO/D pode ser de dois tipos:

\begin{enumerate}
	\item \textbf{Depósito:} a partir de uma solução, adiciona-se uma quantidade $\delta$ ao feromônio correspondente à cada partícula da solução. No caso de um problema em grafos, por exemplo, para cada aresta da solução, incrementa-se em $\delta$ o feromônio daquela aresta.
	\item \textbf{Evaporação:} similar ao depósito, mais ao invés de incrementar o feromônio em $\delta$, decrementa-se.
\end{enumerate}

O depósito de feromônio ocorre quando novas soluções não-dominadas são encontradas. A evaporação no MACO/D, diferentemente da maioria dos algoritmos baseados em ACO, não ocorre para todos os feromônios em todas as iterações. Ao invés disso, o feromônio só é decrementado quando uma solução deixa de ser não-dominada (ver algoritmo \ref{alg_macod_geral}).

Dada uma solução $s$ do problema do roteamento multicast (PRM), se $s$ deve ser reforçada (depósito), cada aresta $e$ da árvore $s$ incrementa o valor correspondente na estrutura de feromônios em um fator $\delta$. Matematicamente, esse processo de depósito é definido por:

\[\delta(e) = (1 - pesos(e)) \times \rho\]

Sendo, $pesos(e)$ é a média dos pesos normalizados na aresta $e$, e $\rho$ é a taxa de evaporação (parâmetro do MACO/D). Se $s$ deve ser desencorajada (evaporação), os valores na matriz de feromônios correspondentes às arestas de $s$ devem ser decrementados em $\delta$.

Dada uma solução $s$ do problema da mochila multiobjetivo (PMM), se $s$ deve ser reforçada (depósito), cada item $i$ de $s$ incrementa o valor correspondente na estrutura de feromônios em um fator $\delta$. O cálculo do depósito no PMM é dado por:

\[\delta(i) = lucros(i) \times (1 - peso(i) / peso\_{max}) \times \rho\]

Onde $lucros(i)$ é a média dos valores normalizados de lucro do item $i$, e $peso\_{max}$ é o maior peso dentre todos os itens. Se $s$ deve ser desencorajada (evaporação), os valores no vetor de feromônios correspondentes aos itens de $s$ devem ser decrementados em $\delta$.

%O fator de incremento de feromônio em ACOs multiobjetivos é normalmente fixo e independente tanto da solução quanto da partícula (aresta ou item). A ideia de se basear a quantidade de feromônios na qualidade da partícula é utilizada em nosso modelo devido aos experimentos realizados mais a frente no texto, na etapa 2 do capítulo \ref{chapter_experimentos} (seção \ref{section_experimentos_etapa2}).