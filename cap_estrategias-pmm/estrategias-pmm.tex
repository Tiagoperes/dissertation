\chapter[Estratégias evolutivas para o PMM]{Estratégias evolutivas para o PMM}

Uma das partes mais importantes de um algoritmo de busca bio-inspirado é a modelagem da solução. Para os algoritmos genéticos é necessário definir a representação da solução, a geração aleatória de indivíduos, o cruzamento e a mutação. Para os ACO's, é possível utilizar a mesma representação de solução do AG, mas deve-se desenvolver um algoritmo que constrói a solução a partir de uma estrutura de feromônios e uma heurística.

A implementação de um AG para o problema da mochila mono-objetivo é trivial, pois a solução é representada por um vetor binário e a literatura está repleta de exemplos que podem ser resolvidos dessa maneira. Um AG para o problema da mochila original pode ser encontrado em \cite{Hristakeva2013}. A versão many-objective do problema não requer nenhuma modificação no modelo, fazendo com que o mesmo processo de cruzamento e mutação possam ser utilizados. No entanto, a implementação de um ACO para o mesmo problema pode ser desafiador, já que as colônias de formigas esperam trabalhar com grafos e não arrays de bits. Um estudo extensivo relativo ao uso de ACO's para a resolução do problema da mochila com ACO's foi feito a partir dos trabalhos \cite{Ke2010,kong2008,changdar2013,Alaya2004,Alaya2007}. Nas seções a seguir detalhe-se a representação da solução, os operadores genéticos e a construção de soluções para o PMM usados neste trabalho.

\section{Representação da solução}
Em ambas as estratégias bio-inspiradas, AG's e ACO's, uma solução para o PMM é representada da mesma maneira: um vetor binário. Se a instância do problema da mochila apresenta 10 itens ao total, por exemplo, um vetor com 10 bits representa a solução. As posições do vetor representam a ausência ou a presença de um item, se a posição $i$ do vetor é 0, então o $i$-ésimo item não faz parte da solução, caso contrário, se o vetor na posição $i$ vale 1, então o  $i$-ésimo item está presente na mochila. E.g. se a solução é representada pelo vetor [1,0,0,1,0,1,1,0,0,0], apenas os itens 0, 3, 5 e 6 serão colocados na mochila, os outros ficarão de fora.

Para gerar uma solução aleatória, basta sortear os valores 0 ou 1 para cada posição do vetor. Tanto a criação aleatória, quanto a combinação de vetores (cruzamento) não garante a formação de uma solução válida, i.e., é possível que se crie um vetor, onde a soma dos pesos dos itens ultrapasse a capacidade da mochila. Portanto, após a geração de qualquer solução, seja aleatória ou proveniente de um \textit{crossover}, é necessário executar um processo de validação da solução. Para validar um vetor binário, basta verificar a soma dos pesos, caso seja maior que a capacidade da mochila, remove-se um item aleatório. Repete-se o processo até que a solução seja válida, ou seja, até que respeite a capacidade da mochila \cite{Ishibuchi2015}.

\section{Cruzamento e mutação (AGs)}
O cruzamento entre duas soluções binárias, como explicado em \ref{section_ag}, pode ser efetuado de diversas maneiras. Neste trabalho, foi utilizado crossover uniforme, ou seja, o filho herda de forma aleatória os bits do pai 1 ou do pai 2. Veja o exemplo da figura \ref{fig_cross_uniforme}.

\begin{figure}[!htbp]
	\label{fig_cross_uniforme}
	\caption{Exemplo de crossover uniforme}
	\centering
	\renewcommand{\arraystretch}{2} 
	\begin{tabular}{rl}
		Pai 1: & 
		\renewcommand{\arraystretch}{1.15} 
		\begin{tabular}{|c|c|c|c|c|c|}
			\hline 
			\rowcolor[HTML]{F5D1CF}
			0 & 0 & 1 & 0 & 1 & 1 \\
			\hline 
		\end{tabular}
		\\
		Pai 2: & 
		\renewcommand{\arraystretch}{1.15} 
		\begin{tabular}{|c|c|c|c|c|c|}
			\hline 
			\rowcolor[HTML]{CCCCFF}
			1 & 0 & 0 & 1 & 1 & 0 \\
			\hline 
		\end{tabular}
		\\
		Máscara: & 
		\renewcommand{\arraystretch}{1.15} 
		\begin{tabular}{|c|c|c|c|c|c|}
			\hline 
			0 & 1 & 0 & 0 & 1 & 0 \\
			\hline 
		\end{tabular}
		\\
		Filho 1: & 
		\renewcommand{\arraystretch}{1.15} 
		\begin{tabular}{|c|c|c|c|c|c|}
			\hline 
			\cellcolor[HTML]{F5D1CF}0 & \cellcolor[HTML]{CCCCFF}0 & \cellcolor[HTML]{F5D1CF}1 & \cellcolor[HTML]{F5D1CF}0 & \cellcolor[HTML]{CCCCFF}1 & \cellcolor[HTML]{F5D1CF}1 \\
			\hline 
		\end{tabular}
		\\
		Filho 2: & 
		\renewcommand{\arraystretch}{1.15} 
		\begin{tabular}{|c|c|c|c|c|c|}
			\hline 
			\cellcolor[HTML]{CCCCFF}1 & \cellcolor[HTML]{F5D1CF}0 & \cellcolor[HTML]{CCCCFF}0 & \cellcolor[HTML]{CCCCFF}1 & \cellcolor[HTML]{F5D1CF}1 & \cellcolor[HTML]{CCCCFF}0 \\
			\hline 
		\end{tabular}
	\end{tabular}
\end{figure}

Como pode ser visto na figura \ref{fig_cross_uniforme}, o crossover uniforme pode ser implementado com uma máscara, que é um vetor aleatório de bits que controla os genes herdados de cada filho. Se o bit na posição $i$ da máscara vale 0, então o filho na posição $i$ herda o valor do pai 1, caso contrário, o pai 2 fornece o valor. Dessa forma, ainda é possível gerar 2 filhos, um com a regra de que o bit 0 da máscara representa o pai 1 e o bit 1 representa o pai 2 (filho 1 na imagem), e outro com a regra inversa (filho 2).

Após o crossover, existe uma chance definida pelo AG de se mutar a solução. A mutação utilizada para o PMM foi o processo mais simples possível para vetores binários, a inversão de bit: sorteia-se uma posição aleatória do vetor, se o valor for 0, troca-se para 1, caso contrário, troca-se para 0.

\section{Construção da solução (ACOs)}
\label{section_estrategias_pmm_aco}
As colônias de formigas foram propostas inicialmente para problemas em grafos, portanto, soa contra-intuitivo utilizá-las para o problema do mochila. Mas, como mostrado em \cite{Ke2010}, não é necessário ter um grafo para se utilizar a técnica, é possível manipular os feromônios de diversas outras formas. Para o PMM, a literatura traz três principais formas de lidar com o feromônio:

\begin{enumerate} 
	\item Depositar feromônios em cada um dos itens. Sempre que se escolher um objeto para compor a solução, incrementa-se a quantidade de feromônios nele presente. Dessa forma, a quantidade de feromônio em cada item representa a preferência para se escolhê-lo em relação aos demais. \cite{Leguizamon1999}.
	\item Criar um grafo direcionado que representa a preferência de se incluir um item $b$ logo após ter incluído um item $a$. Dessa forma, sempre que se escolher um item $a$ e posteriormente um $b$, deposita-se feromônio na aresta $(a,b)$ do grafo. Ao construir uma solução, analisa-se o último item incluído e todas as arestas no grafo que partem dele, o destino com a maior quantidade de feromônios representa o item preferível. \cite{Fidanova2003}.
	\item A terceira estratégia proposta por \cite{Alaya2004} utiliza um conceito semelhante a ideia anterior, mas ao invés de depositar feromônios apenas em pares consecutivos, os deposita para todos os pares de objetos existentes na solução. Por exemplo, se na solução os itens $a$, $d$ e $f$ já foram incluídos na mochila e no passo corrente adiciona-se o item $c$, o depósito de feromônios é feito nas arestas $(a,c)$, $(d,c)$ e $(f,c)$, de forma que o grafo represente a preferência de se escolher um item dado que algum outro item já tenha sido adicionado. Assim, ao construir a solução, deve-se analisar todas as arestas com origem em algum objeto já presente na solução, o destino da aresta com maior quantidade de feromônios representa o item mais desejável.
\end{enumerate}

Neste trabalho utilizou-se a estratégia 1, os feromônios são depositados nos items. Além dos feromônios, outra importante fonte de informação para se construir uma solução no ACO é a heurística, que estima o quão vantajoso é escolher um item em relação a outro. Tomando como inspiração o estudo de \cite{Ke2010}, propôs-se como heurísticas o seguinte modelo de funções:

\begin{itemize}
	\item Para cada objetivo $k$ (lucro do item), cria-se uma função de heurística $h_k$ que recebe dois parâmetros, a capacidade restante ($cr$) e o item que se deseja incluir. A capacidade restante é a diferença entre a capacidade da mochila e a soma dos pesos dos itens que já foram incluídos. A heurística é então dada por: $h_k(item, cr) = lucro_k(item) * (1 - peso(item) / cr)$.
	\item Uma heurística extra é usada exclusivamente para se referir ao peso do item: $h_{peso}(item) = 1 - peso(item) / peso\_maximo$.
\end{itemize}

Logo, o número de heurísticas no PMM é sempre o número de objetivos + 1. Tendo definido a estrutura de feromônios e as heurísticas, cabe ao algoritmo baseado em colônia de formigas construir a solução.