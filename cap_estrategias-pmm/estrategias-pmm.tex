\chapter[Estratégias evolutivas para o PMM]{Estratégias evolutivas para o PMM}

Uma das partes mais importantes de um algoritmo de busca bio-inspirado é a modelagem da solução. Para os algoritmos genéticos é necessário definir a representação da solução, a geração aleatória de indivíduos, o cruzamento e a mutação. Para os ACOs, é possível utilizar a mesma representação de solução do AG, mas deve-se desenvolver um algoritmo que constrói a solução a partir de uma estrutura de feromônios e uma heurística.

A implementação de um AG para o problema da mochila mono-objetivo é trivial, pois a solução é representada por um vetor binário e a literatura está repleta de exemplos que podem ser resolvidos dessa maneira \cite{KnapsackGA,Hristakeva2013}. A versão \textit{many-objective} do problema não requer nenhuma modificação no modelo, fazendo com que o mesmo processo de cruzamento e mutação possam ser utilizados. No entanto, a implementação de um ACO para essa versão do problema pode ser desafiador, já que as colônias de formigas esperam trabalhar com grafos e não vetores de bits. Um estudo extensivo relativo aos trabalhos encontrados na literatura que adotam ACOs para a resolução do problema da mochila foi realizado \cite{Ke2010,kong2008,changdar2013,Alaya2004,Alaya2007}. Nas seções a seguir, detalhe-se a representação da solução, os operadores genéticos e a construção de soluções para o PMM usados neste trabalho.

\section{Representação da solução}
Em ambas as estratégias bio-inspiradas, AGs e ACOs, uma solução para o PMM é representada por um vetor binário. Se a instância do problema da mochila apresenta 10 itens ao total, por exemplo, um vetor com 10 bits representa a solução. As posições do vetor representam a ausência ou a presença de cada item. Se a posição $i$ do vetor é 0, então o $i$-ésimo item não faz parte da solução. Caso contrário, se o vetor na posição $i$ vale 1, então o  $i$-ésimo item está presente na mochila. Por exemplo, em uma solução representada pelo vetor [1,0,0,1,0,1,1,0,0,0], apenas os itens 0, 3, 5 e 6 são colocados na mochila, os outros itens (1, 2, 4, 7, 8 e 9) são descartados.

Nesse cenário, a geração aleatória de uma solução consiste em sortear os valores 0 ou 1 para cada posição do vetor. Tanto a criação aleatória, quanto a combinação de vetores (cruzamento) não garantem a formação de uma solução válida, ou seja, é possível que se crie um vetor, cuja soma dos pesos dos itens ultrapasse a capacidade da mochila. Portanto, após a geração de qualquer solução, seja aleatória ou proveniente de um \textit{crossover}, é necessário executar um processo de validação da solução. Para validar um vetor binário, basta verificar a soma dos pesos associados aos itens representados por 1. Caso esse valor seja maior que a capacidade da mochila, remove-se um item aleatório. Esse processo é repetido sucessivamente, até que a solução se torne válida, ou seja, até que a soma dos pesos dos itens respeite a capacidade da mochila \cite{Ishibuchi2015}.

\section{Cruzamento e mutação (AG)}
O cruzamento entre duas soluções binárias, como explicado na Seção \ref{section_ag}, pode ser efetuado de diversas maneiras. Neste trabalho, foi utilizado \textit{crossover} uniforme, ou seja, o filho herda de forma aleatória os bits do pai 1 ou do pai 2. Esse processo é ilustrado na \autoref{fig_cross_uniforme}.

\begin{figure}[!htbp]
	\centering
	\renewcommand{\arraystretch}{2} 
	\begin{tabular}{rl}
		Pai 1: & 
		\renewcommand{\arraystretch}{1.15} 
		\begin{tabular}{|c|c|c|c|c|c|}
			\hline 
			\rowcolor[HTML]{F5D1CF}
			0 & 0 & 1 & 0 & 1 & 1 \\
			\hline 
		\end{tabular}
		\\
		Pai 2: & 
		\renewcommand{\arraystretch}{1.15} 
		\begin{tabular}{|c|c|c|c|c|c|}
			\hline 
			\rowcolor[HTML]{CCCCFF}
			1 & 0 & 0 & 1 & 1 & 0 \\
			\hline 
		\end{tabular}
		\\
		Máscara: & 
		\renewcommand{\arraystretch}{1.15} 
		\begin{tabular}{|c|c|c|c|c|c|}
			\hline 
			0 & 1 & 0 & 0 & 1 & 0 \\
			\hline 
		\end{tabular}
		\\
		Filho 1: & 
		\renewcommand{\arraystretch}{1.15} 
		\begin{tabular}{|c|c|c|c|c|c|}
			\hline 
			\cellcolor[HTML]{F5D1CF}0 & \cellcolor[HTML]{CCCCFF}0 & \cellcolor[HTML]{F5D1CF}1 & \cellcolor[HTML]{F5D1CF}0 & \cellcolor[HTML]{CCCCFF}1 & \cellcolor[HTML]{F5D1CF}1 \\
			\hline 
		\end{tabular}
		\\
		Filho 2: & 
		\renewcommand{\arraystretch}{1.15} 
		\begin{tabular}{|c|c|c|c|c|c|}
			\hline 
			\cellcolor[HTML]{CCCCFF}1 & \cellcolor[HTML]{F5D1CF}0 & \cellcolor[HTML]{CCCCFF}0 & \cellcolor[HTML]{CCCCFF}1 & \cellcolor[HTML]{F5D1CF}1 & \cellcolor[HTML]{CCCCFF}0 \\
			\hline 
		\end{tabular}
	\end{tabular}
	\caption{\label{fig_cross_uniforme}Exemplo de crossover uniforme}
\end{figure}

Como pode ser visto na figura \ref{fig_cross_uniforme}, o \textit{crossover} uniforme pode ser implementado com uma máscara, que é um vetor aleatório de bits que controla os genes herdados de cada filho. Se o bit na posição $i$ da máscara vale 0, então o gene do filho na posição $i$ herda o valor do pai 1, caso contrário, herda o valor do pai 2. Dessa forma, ainda é possível gerar 2 filhos: um usando a regra na qual o bit 0 da máscara representa o pai 1 e o bit 1 representa o pai 2 (filho 1 na figura); e outro usando os genes complementares (filho 2).

Após o \textit{crossover}, existe uma chance, determinada pela taxa de mutação definida para o AG, do indivíduo sofrer alguma mutação genética. Neste trabalho, foi utilizado o método mais simples de mutação para vetores binários, ou seja, a inversão de bit. Esse processo consiste em sortear uma posição aleatória do vetor e se o valor for 0, troca-se para 1, caso contrário, troca-se para 0.

\section{Construção da solução (ACO)}
\label{section_estrategias_pmm_aco}
As colônias de formigas foram propostas inicialmente para problemas em grafos, e, portanto, não é intuitivo sua adaptação para outros tipos de problemas, como o problema do mochila. Entretanto, como mostrado em \cite{Ke2010}, não é necessário ter um grafo para se utilizar a técnica, sendo possível manipular os feromônios de diversas outras formas. Foram encontradas na literatura três diferentes formas de lidar com os feromônios para o problema da mochila multiobjetivo (PMM). A primeira abordagem consiste em depositar feromônios em cada um dos itens \cite{Leguizamon1999}. Sempre que um item é escolhido para compor a solução, incrementa-se a quantidade de feromônios presente no item. Dessa forma, a quantidade de feromônio em cada item representa sua preferência em relação aos demais. A segunda abordagem propõe a criação de um grafo direcionado que representa a preferência em incluir um item $b$ após a inclusão do item $a$ \cite{Fidanova2003}. Dessa forma, sempre que se escolher um item $a$ e posteriormente um $b$, deposita-se feromônio na aresta $(a,b)$ do grafo. Ao construir uma solução, analisa-se o último item selecionado e todas as arestas no grafo que partem dele. O destino com a maior quantidade de feromônios representa o item preferível. A terceira estratégia foi proposta por \cite{Alaya2004} utiliza um conceito semelhante a abordagem anterior. Entretanto, ao invés de depositar feromônios apenas em pares consecutivos, eles são depositados para todos os pares de objetos existentes na solução. Por exemplo, se na solução os itens $a$, $d$ e $f$ estão na mochila e no passo atual adiciona-se o item $c$, o depósito de feromônios é feito nas arestas $(a,c)$, $(d,c)$ e $(f,c)$, de forma que o grafo represente a preferência de se escolher um item dado que algum outro item já tenha sido selecionado. Assim, ao construir a solução, deve-se analisar todas as arestas com origem em algum item já presente na solução, o destino da aresta com maior quantidade de feromônios representa o item mais desejável. Neste trabalho utilizou-se a primeira estratégia que deposita os feromônios diretamente nos itens.

Além dos feromônios, outra importante fonte de informação para se construir uma solução no ACO é a heurística. Ela é responsável por estimar o quão vantajoso é escolher um item em relação a outro. Tomando como inspiração o estudo de \cite{Ke2010}, nossas implementações do ACO utilizam como heurísticas o seguinte modelo de funções:

\begin{itemize}
	\item Para cada objetivo $k$ (lucro do item), cria-se uma função de heurística $h_k$ que recebe dois parâmetros, a capacidade restante ($cr$) e o item que se deseja incluir. A capacidade restante é a diferença entre a capacidade da mochila e a soma dos pesos dos itens que já foram incluídos. A heurística é então dada por: \[h_k(item, cr) = lucro_k(item) \times (1 - \frac{peso(item)}{cr})\].
	\item Uma heurística extra é usada exclusivamente para se referir ao peso do item, a qual é dada por: \[h_{peso}(item) = 1 - \frac{peso(item)}{peso\_maximo}\].
\end{itemize}

Logo, o número de heurísticas no PMM é sempre o número de objetivos + 1. Tendo definido a estrutura de feromônios e as heurísticas, para construir uma solução, o algoritmo recebe a função heurística ($h$) e um vetor de feromônios ($\tau$). Inicialmente, a lista de exploração $Exp$ contém todos os itens possíveis, a probabilidade $p$ de cada item $exp_i \in Exp$ é calculada de acordo com a heurística de $exp_i$ ($h(exp_i)$) e a quantidade de feromônios no item $tau_i$, como na fórmula seguinte:

\[p(exp_i) = \frac{\tau_i^\alpha + h(exp_i)^\beta}{\sum_{j \gets 0}^{tamanho(Exp)} \tau_j^\alpha + h(exp_j)^\beta}\]

O próximo item na solução é escolhido de acordo com as probabilidades calculadas em uma espécie de roleta, onde, quanto maior o valor de probabilidade ($p(exp_i)$), maior a chance do item ser escolhido. Após ter selecionado um item, recalcula-se a lista $Exp$ para que contenha apenas itens válidos para a solução, isto é, remove-se qualquer elemento de $Exp$ que, se incluído na solução, excede a capacidade da mochila. O processo é repetido até que $Exp$ seja uma lista vazia.

A fim de reduzir a complexidade do algoritmo de construção da solução para o PMM, ao invés de se utilizar a lista completa de itens possíveis ($Exp$), utiliza-se uma amostra desse conjunto ($Exp'$). $Exp$ é calculado normalmente, mas ao submeter-se o conjunto para o cálculo de probabilidades e roleta, no lugar de $Exp$, é utilizado $Exp'$, uma amostra aleatória de tamanho fixo $Exp'_{size}$ (parâmetro do algoritmo) da lista completa $Exp$.
