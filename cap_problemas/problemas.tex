\chapter[Problemas de teste]{Problemas de teste}

São vários os problemas em que se pode aplicar os algoritmos multiobjetivos e pode-se dividi-los em duas categorias: contínuos ou discretos. O comportamento e até a própria possibilidade de se aplicar o algoritmo depende dessa classificação. A fim de avaliar os métodos multiobjetivos, normalmente se utiliza problemas de teste, dentre os contínuos, destacam-se: SCH, FON, POL, KUR e ZDT, todos podem ser encontrados no artigo original do NSGA-II \cite{Deb2002}. Os problemas contínuos são funções contínuas e não necessariamente representam um problema real. Os problemas discretos, por outro lado, possuem um enunciado bem definido e nem todas as soluções possíveis são válidas, ou seja, existem buracos no contradomínio das funções. Exemplos de problemas discretos comummente usados na literatura multiobjetivo são: cacheiro viajante, roteamento de veículos com janelas de tempo, problema da mochila, sequenciamento de proteínas e problemas de roteamento em redes. Neste trabalho focou-se em dois problemas discretos: o problema da mochila multiobjetivo (PMM) e o problema do roteamento multicast (PRM).

\section{Problema da mochila multiobjetivo}

O \ac{PM} é um problema teórico muito conhecido na computação e geralmente utilizado para se introduzir o conceito de otimização. Apesar disso, existem problemas reais equivalentes que podem ser resolvidos com as mesmas técnicas, como o escalonamento de tarefas em um sistema operacional.

O problema da mochila consiste em arranjar um conjunto de itens em uma mochila de forma a não exceder a capacidade da mesma e ao mesmo tempo maximizar o valor (lucro) dos objetos carregados. Matematicamente, dado uma mochila de capacidade $C$ e um conjunto de itens $O$, onde cada $O_i \in O$ possui um peso $peso(O_i)$ e um lucro $lucro(O_i)$, encontrar o conjunto $S \subset O$, tal que $\sum_{o \in S} peso(o) \leq C$ e $\sum_{o \in S} lucro(o)$ seja o maior possível.

Existem diversas estratégias para se resolver o \ac{PM}, dentre elas as mais usadas são os algoritmos gulosos e os algoritmos genéticos. Uma coletânea de algoritmos gulosos para o \ac{PM} são explicados, implementados e analisados em \cite{Martello1990}.  Em \cite{Khuri1994}, um AG é proposto, mostrando o potencial da estratégia para a resolução de problemas de otimização NP-completos com restrições. Apesar de existirem múltiplas estratégias, os algoritmos gulosos são os mais rápidos e eficientes para resolver o PM mono-objetivo, em contra-partida, a complexidade adicionada pela versão multiobjetivo (PMM) inviabiliza a utilização dos mesmos, tornando os AG's e demais métodos bio-inspirados as melhores opções.

O problema da mochila multiobjetivo (PMM) é similar ao original, sua única diferença está no fato de que cada item, ao invés de possuir um único valor (lucro), é composto de múltiplos valores. No PMM, a função $lucro(O_i)$ retorna um vetor ao invés de um escalar, onde cada componente representa o valor do item $O_i$ em um dos objetivos. Por exemplo, no PMM de 3 objetivos, cada $O_i \in O$ possui um vetor tri-dimensional de lucros. O objetivo do problema passa a ser maximizar todos os lucros ao invés de um único valor.

O PMM já foi utilizado várias vezes para avaliar algoritmos multi-objetivos, podendo-se destacar os trabalhos de \cite{Zitzler1999}, \cite{Zitzler2002} e \cite{Zhang2007}. As instância do PMM utilizadas no decorrer deste trabalho foram geradas de forma aleatória e compreendem problemas da mochila com 30, 40, 50, 100 e 200 itens. Para gerar as instâncias, sorteou-se valores no intervalo (0, 1000) para os lucros e para os pesos. A capacidade da mochila foi sempre definida como 60\% da soma dos pesos.  

\section{Problema do roteamento multicast (PRM)}
\label{section_problemas_prm}

O problema do roteamento multicast aparece na engenharia de tráfego em redes de computadores e consiste em transmitir uma mensagem multicast. Uma transmissão de rede pode ser do tipo unicast, multicast ou broadcast. Em transmissões unicast conecta-se um ponto da rede a um outro ponto qualquer, para fazer isso de forma eficiente basta encontrar o melhor caminho entre os dois pontos. As comunicações broadcast caracterizam-se pelo fato de um nó da rede (servidor) enviar o conteúdo a todos os demais, para obter as melhores rotas para trafegar os dados, basta verificar a árvore geradora de custo mínimo. As transmissões multicast desejam, a partir de um nó da rede, transmitir o conteúdo para alguns outros, o que apresenta maior complexidade, pois é necessário obter uma árvore de Steiner de custo mínimo, o que é mais difícil que calcular uma única rota ou construir a árvore geradora de custo mínimo \cite{Bueno2010}.

O PRM, além de ser um problema prático, é muito importante, pois significaria um grande avanço na geração de rotas em redes de computadores, proporcionando uma comunicação mais rápida, menos custosa e mais confiável entre dispositivos, o que é essencial em uma era onde a maioria das pessoas consomem informação e entretenimento pela Internet. Dado que deseja-se transmitir um conteúdo via uma rede de computadores, o problema consiste em encontrar a melhor rota possível entre a fonte de dados e os destinos. Matematicamente, dado uma rede representada pelo grafo $G=(V,E)$, um nó raiz $s \in V$ (nó transmissor) e um conjunto de nós destinos $D \subset V$ (nós receptores), o PRM consiste em determinar a subárvore $T$ de $G$ enraizada em $r$ que inclui todos os vértices em $D$ e apresenta o menor custo possível. Veja o exemplo da figura \ref{fig_prm_mono}.

\begin{figure}
	\label{fig_prm_grafo}
	\caption{Exemplo de rede retirado de \cite{Bueno2010}}
	\centering
	\includegraphics[width=1\textwidth]{cap_problemas/figs/prm_grafo}
\end{figure}

\begin{figure}
	\label{fig_prm_mono}
	\caption{Exemplos de árvores multicast relativos ao grafo da figura \ref{fig_prm_grafo}. Retirado do trabalho de \cite{Lafeta2016}}
	\centering
	\includegraphics[width=0.8\textwidth]{cap_problemas/figs/prm_mono}
\end{figure}

Na figura \ref{fig_prm_mono} apresenta-se exemplos de árvores multicast criadas a partir do grafo mostrado na figura \ref{fig_prm_grafo} considerando a raiz ($r$) como sendo o vértice 0 e os nós destinos ($D$) igual a {1, 8, 12, 13}. O custo de cada árvore é dado pela soma dos custos de suas arestas, dentre os exemplos, a árvore mais à direita possui o menor custo total: 65.

O PRM original é proposto com apenas um objetivo a se otimizar, mas o intuito deste trabalho é utilizar uma versão mais realista do problema. A qualidade de um enlace de rede não pode ser medida através de uma única métrica, um custo genérico não é capaz de dizer se um link é bom ou ruim, características como distância, delay, capacidade de tráfego e uso do tráfego são melhores indicadores, portanto, propõe-se como objeto de estudo deste trabalho, o problema do roteamento multicast multiobjetivo. Nesta versão do problema, as árvores apresentadas como solução devem representar o melhor compromisso entre as métricas utilizadas. Veja um exemplo para a otimização de ``custo'' e ``delay'' na figura \ref{fig_prm_multi}.

\begin{figure}
	\label{fig_prm_multi}
	\caption{Exemplo de árvore multicast no PRM multiobjetivo}
	\centering
	\includegraphics[width=1\textwidth]{cap_problemas/figs/prm_multi}
\end{figure}

Na figura \ref{fig_prm_multi} apresenta-se um exemplo de rede com as métricas ``custo'' (primeiro valor) e ``delay'' (segundo valor) nas arestas. As arestas em negrito representam uma árvore multicast ótima (não-dominada) para o seguinte conjunto de objetivos:

\begin{enumerate} 
	\item Minimizar custo total: soma dos valores de custo para cada aresta da árvore;
	\item Maximizar delay fim-a-fim atendidos: número de ramos da árvore em que a soma dos delays nas arestas não ultrapassa um valor $d_{max}$ pré-definido, neste caso 25. Em outras palavras, quantidade de conexões cliente-servidor que mantém limite aceitável de atraso.
\end{enumerate}

Neste trabalho considera-se até quatro valores de peso para um enlace rede: custo, \textit{delay}, capacidade de tráfego e tráfego corrente, representados nas fórmulas a seguir respectivamente pelas funções: $c()$, $d()$, $z()$ e $t()$. Através dessas medidas são formulados os seguintes objetivos:

\begin{enumerate} 
	\item Custo total: soma dos valores de custo para cada aresta da árvore;
	\item Delay fim-a-fim médio: média da soma dos \textit{delays} em cada ramo da árvore. Em outras palavras, média do atraso em cada uma das comunicações cliente-servidor;
	\item Delay fim-a-fim máximo: maior valor para a soma de \textit{delays} dentre todos os ramos da árvore;
	\item \textit{Hops count}: número de vértices na árvore;
	\item Utilização máxima de enlaces: considerando todas as arestas na árvore, qual delas atinge a maior utilização de banda? Matematicamente, considerando $E$ o conjunto de arestas da árvore e $\phi$ o tamanho da mensagem, $\max_{e \in E} \frac{t(e) + \phi}{z(e)}$;
	\item Utilização média dos enlaces: média entre a utilização de banda entre todas as arestas da árvore. Similar à definição anterior.
\end{enumerate}

Afim de possibilitar diversos cenários de teste para o PRM, os objetivos acima podem ser combinados de diversas maneiras, criando vários ambientes multi-objetivos. Neste trabalho foram utilizadas 5 formulações de objetivo:

\begin{enumerate}
	\item $P_2$: formado pelos objetivos 1 e 3;
	\item $P_3$: formado pelos objetivos 1, 3 e 4;
	\item $P_4$: formado pelos objetivos 1, 3, 4 e 5;
	\item $P_5$: formado pelos objetivos 1, 3, 4, 5 e 6;
	\item $P_6$: formado pelos objetivos 1, 2, 3, 4, 5 e 6.
\end{enumerate}

O PRM foi trabalhado sobre 5 redes diferentes variando a complexidade em termos de quantidade de nós destinos, vértices e arestas. Essas redes foram retiradas do trabalho de \cite{Lafeta2016} e suas caraterísticas são apresentadas na tabela \ref{tab_prm_redes}.

\begin{table}[!htbp]
	\centering
	\caption{Definições das redes utilizadas pelo PRM}
	\label{tab_prm_redes}
	\begin{tabular}{r|rrr}
		Nome        & Destinos & Vértices & Arestas \\ \hline
		Rede 1 (R1) & 10       & 33       & 106     \\
		Rede 2 (R2) & 18       & 75       & 188     \\
		Rede 3 (R3) & 37       & 75       & 188     \\
		Rede 4 (R5) & 12       & 75       & 300     \\
		Rede 5 (R5) & 16       & 100      & 250     \\ \hline
	\end{tabular}
\end{table}
